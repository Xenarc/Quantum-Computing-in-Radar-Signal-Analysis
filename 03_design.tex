\section{Research Design \& Methodology}~\label{sec:design}

\subsection{Research Methodology}

To achieve the objective of \textbf{acquiring radar signal test data}, \todojc{Free and open access emulator / generator perhaps, is your preference not the necessity, may be skip "free and open access" your solution could secure such a device. 
Also another approach still is to acquire some data set, which you may need to acknowledge here.}\textbf{a free and open-access signal emulator} will need to be acquired and validated. 
It should have capability of  generating a set of radar signals exhibiting various degrees of signal quality and \todojc{You may need to explain what are "env situations"}\textbf{environmental situations}. 
The variable signal quality refers to a signal-to-noise ratio, and environmental conditions refer to number of radars, pulse density, and pulse frequency.

\todojc{Note: it is easier to manage and review latex texts when paragraphs sentences are written on separate lines}To achieve the second objective: "To determine a suitable method of encoding time-domain radar signals into quantum form." an experimental study will be conducted. 
The independent variable being the type of quantum encoding algorithm used to input the radar signals. 
The dependent variable will be the precision, computational efficiency, and qualitative applicability of the quantum method for use in subsequent experimentation. 
One approach shall be selected as a baseline for the next steps. 
Specifically, this will involve designing and implementing various quantum algorithms for encoding time-series signals, testing them on a set of radar signals, under various degrees of signal quality and environmental situations, and collecting data on their suitability. 
The results should be tabulated by method versus suitability under each condition (defined above).

To determine a suitable quantum method for pulse detection, an exploratory approach will be adopted. Suitable, in this context being a method which is quantitatively measurable in its pulse detection performance. 
A comparison of actual and computed pulse boundaries should be made. 
If no performance can be measured, an analysis of why should follow. 
Since no known quantum algorithms exist for pulse detection, the concrete method will be to develop a pulse-detection algorithm and document the process of iteration. 
Such documentation on each iteration may be the quantitative performance, a qualitative performance evaluation. 
The \todojc{Rephrase}\textbf{definition of done} is whether or not a reasonable quantitative measure of performance \todojc{Can be achieved?}\textbf{can be taken}. Input data may be varied to better understand the suitability of the method.

Thirdly, extracting signal parameters may be undertaken in a similar exploratory procedure as the prior - to document the iteration of the algorithms, and present a quantitative measure of performance. Such measures being the accuracy, precision, and computational efficiency of the quantum method. Input data may be chosen as either pre-prepared pulse-only data, raw signal data, or a combination thereof, under those signal quality and environmental conditions previously noted. A similar process of iterative documentation is to be made, with quantitative performance or qualitative performance evaluations for each iteration.
\todojc{I understand this is all preliminary, however, you need to be more precise and indicate that an experimental methodology is to be adopted (why and ref), what kind of experiments are to be conducted (how and ref), provide some details of such experiments (you have done this informally), what kind of data is going to be collected (how and in what format), and then how the collected data is going to be analysed (and presented).}
\todojc{And yes, you will need to reference some paper / book chapters on the adopted methodology, which may also guide you in research design}
% Objective 4: To determine possible quantum methods for de-interleaving radar signals.
% Signal clustering / de-interleaving : either after having done descretised a signal into PDW's, or for more unrefined signals where an amorphous 'similarity' between continuous times and domains is drawn. The measure of success would be: Successfully cluster or de-interleave a signal into PDWs (objective is accuracy); or for more unrefined signals / encoded IQ, a Boolean feasibility of whether a continuous measure of 'similarity' between signals can be drawn.

% Experimental Design:








