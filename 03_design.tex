
% C. Wohlin, P. Runeson, M. H ̈ost, M. C. Ohlsson, B. Regnell, and A. Wessl ́en. Experimentation in Software Engineering. Springer Science & Business Media.
% M. Oivo, P. Kuvaja, P. Pulli, and J. Simil ̈a. Software engineering research strategy: Combining experimental and explorative research (eer). pages 302–317. Springer


\section{Research Design \& Methodology}~\label{sec:design}

In this section will be discussed the various research methods and techniques that will be used in this study.
First is the research methodology, wherein an examination of approaches will be made.
Secondly, the data environment will be defined, along with the controlled variables.
Finally the research design will be presented, followed by a discussion of the data collection methods, data analysis techniques, and methods for ensuring the validity the findings.

\todojc{The chapter should include two parts, i.e. the overall research methodology with justification (from the research question and objectives) and the details of your research design (what steps are to be taken as part of your research). The latter can be visualised as a chart of what are the research phases (which may include what would happen in T1 and then in T2) with the some description and justification. Most likely this would relate to your research objectives derived from your research question.}
\subsection{Research Methodology}

%%% The methodological framework

% RQ: (but different!)
This investigation aims to determine how can quantum computing methods be used to in an \ac{ESM} function to detect the types and characteristics of radar signals in the defence context.
% Partitio
Three aspects of this question will be explored: quantum encoding, pulse detection, and frequency estimation algorithms.
% Various research methodologies may be employed
Various research methodologies may be used to answer this question, including experimental, surveys, observational, and case study approaches.

% Experimental
\todojc{YOu need some refs for support here}Experimental research appears to be the most suitable methodology as it allows for well-defined set of inputs and dependents, that ultimately result in objective and precise data that can be analyzed statistically.
This enables a more indicative identification of the true nature of the quantum algorithms under test, thereby providing a deeper understanding of the solution itself.
In this respect, an experimental approach more truthfully validates the performance and consequent suitability of these new methods, thus increasing the reliability and generalisability of the results.
Furthermore, by systematically testing various scenarios, the experimental methodology attacks the problem from multiple aspects, thereby allowing for the discovery of new patterns and unexpected results - more than what could be achieved by any other of the alternative approaches. 

% Rebuttal
\todojc{I have rephrased this}As there are few attempts at using quantum methods in radar signal processing, survey methodology, while useful for accumulating existing domain expertise (in the form of attitudes, beliefs, and opinions), cannot adequately and concretely examine the feasibility of such an unexplored approach to \ac{ESM}.
In other words, it is only by knowing a quantum algorithm that, through study, its nature can be understood.
Similarly, the scant availability of research on the topic, as revealed by the literature review, renders observational research methodology unsuitable.
Finally, unlike experimental approaches, a case study methodology is less generalisable and replicable.

% .: Experimental
For these reasons, the methodological framework adopted, for the three aspects of the research question, is experimental. 
% Describe basili
The approach adopted will be that of Basili et al. \cite{basili_experimentation_1985}.


% No hypothesis
\todojc{You need some refs on confirmatory vs exploratory research}Given little accessible research on the combination of quantum computing and \ac{ESM} signal analysis, this study will not present any \`{a} priori hypotheses, as there is not enough evidence to make a considered judgement.
% .: Exploratory
This is therefore an exploratory enquiry, intended to understand the feasibility of these quantum techniques and identify areas for further investigation, in contrast to the alternative confirmatory approach.
It is also inductive, in the respect that the study will derive findings from data gathered during the research process rather than from an existing theoretical framework for quantum \ac{ESM} analysis.

\todojc{Nice tables but the reader will not understand what they mean - you need to discuss them in text - note that tables and figures do not speak for themselves - you need to help them!}According to the experimental framework, the \textit{definition} of the experiment is defined in Table \ref{tab:exp_definition}, \textit{planning} in Table \ref{tab:exp_planning}, and \textit{operation} in Table \ref{tab:exp_operation}.

\begin{table}[ht]
\caption{Experimental definition}
\label{tab:exp_definition}
\begin{tabular}{p{0.16\linewidth}|p{0.28\linewidth}p{0.28\linewidth}|p{0.28\linewidth}}
\hline
& Experiment 1: Quantum encoding & Experiment 2: Pulse detection & Experiment 3: frequency estimation \\
\hline
Motivation & To assess & To understand & To understand \\
Object & Pulsed-Doppler radar & \ac{CW} radar & \ac{FMCW} radar, pulsed \ac{CW} \\
Purpose & so that their performance and nature may be understood & so that pulses can be associated to individual radars & so that the frequency of a pulse can be extracted and used for further   analysis \\
Perspective & Researcher & Researcher & Researcher \\
Domain & Quantum encoding methods for ESM signals & Quantum pulse detection algorithm & Quantum frequency algorithm \\
Scope & Multi-project & Single project & Single project \\
\hline
\end{tabular}
\end{table}

\begin{table}[ht]
\caption{Experiment Planning}
\label{tab:exp_planning}
\begin{tabular}{p{0.16\linewidth}|p{0.28\linewidth}p{0.28\linewidth}|p{0.28\linewidth}}
\hline
& Experiment 1: Quantum encoding & Experiment 2: Pulse detection & Experiment 3: frequency estimation \\
\hline
Design & Randomised block & Completely randomised & Completely randomised \\
Criteria & the quality and suitability of the method & An accurate  indication of pulse edge & An accurate frequency indication given some encoded signal input \\
Measurement & circuit size, expressibility, sampling capacity, bandwidth, and   computational efficiency & precision, recall, and F1 metrics & \ac{RMSE} frequency error, aggregated over all samples. \\
\hline
\end{tabular}
\end{table}

\begin{table}[ht]
\caption{Experiment Operation}
\label{tab:exp_operation}
\begin{tabular}{p{0.16\linewidth}|p{0.28\linewidth}p{0.28\linewidth}|p{0.28\linewidth}}
\hline
& Experiment 1: Quantum encoding & Experiment 2: Pulse detection & Experiment 3: frequency estimation \\
\hline
Preparation & Pilot study & Pilot study & Pilot study \\
Execution & Collection and validation Automated by GNU radio & Collection and validation Automated by GNU radio & Collection and validation Automated by GNU radio \\
Analysis & Plots of measure vs encoding model &
\begin{itemize}
    \item data-table of metrics
    \item Histogram of measured pulse boundaries
    \item Scatter plot of measured vs actual pulse boundaries
\end{itemize}
&
\begin{itemize}
    \item data-table of metrics
    \item Histogram of measured results at different frequencies
    \item Satter plot of measured vs actual frequency
\end{itemize}\\
\hline
\end{tabular}
\end{table}

\subsubsection{The data environment}

% Controlled and simulated
A choice of simulated conditions arose by necessity, since acquiring sample radar systems is impractical, cost prohibitive, and more importantly - out of scope.
It was therefore pertinent that a radar signal simulator be acquired.

%%% Experimental Partitio
To address each aspect of the research question - encoding, detection, and frequency estimation - a unique experimental design is required that takes into account the particular characteristics of the problem.
%%% General Criteria
In each case, the method's performance will be evaluated using quantitative measures.
\todojc{I think "how well" is needed and your objectives identify the need for evaluation, which is "how well"}As a whole, this research is to understand \textit{how} these methods may be used, but not \textit{how well}.
It is therefore more an enquiry into method, rather than a normative study of performance; not, for example, a comparison with classical methods.

%%% Sample nature and collection 
% Describe the nature sample (source data)
\todojc{All these seem very unorganised pieces of facts, which signal type if neeed where?}Each of the approaches perform different functions and therefore require different input data, but in general, the inputs are point source radar signals \cite{chakravorty_what_2018}.
Assumed to have been measured from the operating environment, these signals are simulated and sampled as input to each experiment.
Qualities of these signals are particular to the combination of source radars themselves, however in general, each sample is a complex number.
Complex data types are often employed in classical signal processing algorithms to capture phase and magnitude information of the radar signals, which are typically characterized by a sinusoidal nature.
For these reasons, the interface to the quantum implementations will be complex samples, varying in time.
The generated samples have a resolution of \(64\) bits: \(32\) bit wide floating point numbers for each real and imaginary coefficients.

% Data Collection - how is it collected?
Simulating the signals was decided to be completed using a software package titled 'GNU Radio' \cite{gnu_radio_contributors_gnu_2022}.
Allowing for the simulation of raw signals, the software provides a suite of signal generation, processing, and display blocks which can be used for data collection, quantum experiments, and measurement respectfully.
The software does not come with pre-defined radar models, however, they are to modelled as part of the data preparation process.
% Types of radar
For all experiments, there are three types of radar model that will be simulated (see appendix for GNU-Radio radar model implementations):
\begin{itemize}
    \item Pulsed
    \item Pulsed-Doppler
    \item \ac{FMCW}
\end{itemize}
It is noted that while many other radar types exist, the scope of this preliminary enquiry limits the examination to a smaller selection.
% Controlled radar parameters
A further variable to be controlled, is the regularity of the radar pulses - they shall operate one one frequency (or centre frequency), and at a fixed \ac{PRI}.
The absolute frequency, or centre frequency for modulated signals, is not of particular importance because periodic signals are locally invariant.
Arbitrarily, the frequency \ac{BW} of examination will be \(100MHz\ \pm 50MHz\)
All amplitudes are normalised to \(1\) unless otherwise noted.
Pulsed radar types will assume a rectangular signal envelope: \(0s\) rise/fall time with instant settling time.
% Noise
Sampled signals will also have a layer of Gaussian noise, representing environmental noise and system noise \todo{validate this as a good analogue for real-world}.
The signal-to-noise ratio, for a signal with amplitude \(1\) will be set at \todo{Define this}\textbf{\(X\)} for all experiments.

% Sample rate definition and justification
GNU-Radio, as with any sampled signal processors, requires a sample rate.
Following the cardinal theorem of interpolation \cite{nyquist_certain_1928}, the maximum system frequency of \(150MHz\) implies a Nyquist frequency of \(300MHz\).
The sampling rate of \(300MS/s\) is thus used in all experiments.
Due to the the sample rate constraint, a built-in \ac{FIR} \ac{LPF} is used to remove all frequencies higher than the Niquist frequency for all generated sample data used in the experiments.

% Is the collection valid? (Procedures used to ensure data quality)
Simulated radar signals as described here, are reliant on the validity of the radar model being used.
Given that only trivial models are chosen, this risk is small, but not completely eliminated.
\todo{probably want a bit more here...}

\subsubsection{The software environment}

\todojc{I think we started here a new sub-section, header has been added}
% Describe the nature of the approach implementations
For the quantum methods to be implemented on simulated quantum hardware, an instruction set and simulator need be obtained. 
While several software packages fulfil this purpose, the Python-based quantum library \textit{Qiskit} \cite{qiskit_contributors_qiskit_2023} was chosen due to it being free, open-source, and generally familiar to the author and broader research community \cite{garhwal_quantum_2021}.
Furthermore, if it is desired that in future these methods be reproduced on real quantum machines, Qiskit has facilities for doing so.

\todojc{I do not understand this statement}Now that the empirical research framework has been established and the experimental objects, criteria, and materials have been described, each experimental method will be formulated.
