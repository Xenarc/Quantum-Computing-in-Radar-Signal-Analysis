\section{Research Design \& Methodology}~\label{sec:design}
\subsection{Research Methodology}

% C. Wohlin, P. Runeson, M. H ̈ost, M. C. Ohlsson, B. Regnell, and A. Wessl ́en. Experimentation in Software Engineering. Springer Science & Business Media. 
% M. Oivo, P. Kuvaja, P. Pulli, and J. Simil ̈a. Software engineering research strategy: Combining experimental and explorative research (eer). pages 302–317. Springer

%%% The methodological framework
% Motivation
\todojc{Probably not worth stating the obvious}Addressing the research question fully, necessitates a rigorous and validated empirical research framework \cite{basili_experimentation_1985}.
% State the methodological framework
\todojc{You need to argue your case from the research question and its objectives, so say "as ..., therefore the methodological framework needs to be ...", then explain why "exploratory" and what it means, and why not "confirmatory"}The methodological framework adopted will be exploratory in nature and will manifest as a set of experiments to be conducted in sequence.
%%% Explain the methodological framework
% Explain experimental
\todojc{All these are too generic, just state what experiments should aim for in concrete, how many and how}Each experiment will involve manipulating independent variables, which represent the inputs to the quantum approach, to investigate the effects on the dependent variable, representing the output of the quantum method.
% Explain flexible artefact
The approaches themselves will not be pre-determined, and instead will be discovered through the course of experiment.
% Explain controlled variables
Controlled and simulated conditions will be maintained - for both sample data and the implementation of quantum methods - with those algorithms being run on simulated quantum hardware.
% Explain no hypothesis / exploratory
In the absence of an \`{a} priori hypotheses, this exploratory research is intended to understand the performance of these novel quantum techniques and identify areas for further investigation.

%%% Justify this framework:
% Experimental
An \todojc{Ref, we know that we need experiments but what exactly is "experimental methodology" and what it means for the project, specifically!}\textbf{experimental methodology} was chosen to provide a rigorous evaluation of the processing approaches under varying input conditions, allowing a high level of control over the variables in question.
In this respect, an experimental approach more truthfully validates the performance and consequent suitability of these new methods, thereby increasing the reliability and generalisability of the results.
% Exploratory / flexible artefact
As previously noted, little research has been conducted into quantum radar signal processing, it therefore was chosen that a somewhat flexible research methodology \cite{anastas_research_1999} be adopted, where the the objects of study (quantum processing approaches) are not fixed at the time of experimental design.
\textbf{This approach} effectively shortens the feedback loop in developing the quantum methods, ultimately enhancing the value and impact of the results. \cite{karahasanovic_collecting_2005}.\todojc{Very convoluted and I do not understand what it means - we need to discuss}

% Controlled and simulated
A choice of simulated conditions arose by necessity, since acquiring sample radar systems is impractical, cost prohibitive, and more importantly - out of scope.
It was therefore pertinent that a radar signal simulator be acquired.
\todojc{Why simulated hardware? Why not real, we may get access to? IS it important at this stage to know?}Determining that \textbf{simulated quantum hardware} be used over \todo{define in literature review}\textbf{noisy quantum machines} was more challenging. 
Because any quantum solution is naturally bounded by the constraints of the quantum machine, the suitability of these methods is also thus constrained.
\todojc{Not the best argument, as we know that for problems of certain complexity classical methods, and this include quantum simulation, cannot compute the solutions - let's discuss}\textbf{It was for this reason, that simulated hardware was chosen.}
For, given the fact that simulated machines are generally lower in noise and higher in precision, it was deemed they demonstrate a more true and analytic representation of the future potential of these methods.
It is, however, noted as a limitation of this experimental design. If these methods were practically implemented on noisy quantum machines, the results would differ. 

%%% Experimental Partitio
To address each aspect of the research question - encoding, detection, and frequency estimation - a unique experimental design is required that takes into account the particular characteristics of the problem.
%%% General Criteria
In each case, the method's performance will be evaluated using quantitative measures.
As a whole, this research is to understand \textit{how} these methods may be used, but not \textit{how well}.
It is therefore more an enquiry into method, rather than a normative study of performance; not, for example, a comparison with classical methods.

%%% Sample nature and collection 
% Describe the nature sample (source data)
Each of the approaches perform different functions and therefore require different input data, but in general, the inputs are point source radar signals \cite{chakravorty_what_2018}.
Assumed to have been measured from the operating environment, these signals are simulated and sampled as input to each experiment.
Qualities of these signals are particular to the combination of source radars themselves, however in general, each sample is a complex number.
Complex data types are often employed in classical signal processing algorithms to capture phase and magnitude information of the radar signals, which are typically characterized by a sinusoidal nature.
For these reasons, the interface to the quantum implementations will be complex samples, varying in time.
The generated samples have a resolution of \(64\) bits: \(32\) bit wide floating point numbers for each real and imaginary coefficients.

% Data Collection - how is it collected?
Simulating the signals was decided to be completed using a software package titled 'GNU Radio' \cite{gnu_radio_contributors_gnu_2022}.
Allowing for the simulation of raw signals, the software provides a suite of signal generation, processing, and display blocks which can be used for data collection, quantum experiments, and measurement respectfully.
The software does not come with pre-defined radar models, however, they are to modelled as part of the data preparation process.
% Types of radar
For all experiments, there are three types of radar model that will be simulated (see appendix for GNU-Radio radar model implementations):
\begin{itemize}
    \item Pulsed
    \item Pulsed-Doppler
    \item \ac{FMCW}
\end{itemize}
It is noted that while many other radar types exist, the scope of this preliminary enquiry limits the examination to a smaller selection.
% Controlled radar parameters
A further variable to be controlled, is the regularity of the radar pulses - they shall operate one one frequency (or centre frequency), and at a fixed \ac{PRI}.
The absolute frequency, or centre frequency for modulated signals, is not of particular importance because periodic signals are locally invariant.
Arbitrarily, the frequency \ac{BW} of examination will be \(100MHz\ \pm 50MHz\)
All amplitudes are normalised to \(1\) unless otherwise noted.
Pulsed radar types will assume a rectangular signal envelope: \(0s\) rise/fall time with instant settling time.
% Noise
Sampled signals will also have a layer of Gaussian noise, representing environmental noise and system noise \todo{validate this as a good analogue for real-world}.
The signal-to-noise ratio, for a signal with amplitude \(1\) will be set at \todo{Define this}\textbf{\(X\)} for all experiments.

% Sample rate definition and justification
GNU-Radio, as with any sampled signal processors, requires a sample rate.
Following the cardinal theorem of interpolation \cite{nyquist_certain_1928}, the maximum system frequency of \(150MHz\) implies a Nyquist frequency of \(300MHz\).
The sampling rate of \(300MS/s\) is thus used in all experiments.
Due to the the sample rate constraint, a built-in \ac{FIR} \ac{LPF} is used to remove all frequencies higher than the Niquist frequency for all generated sample data used in the experiments.

% Is the collection valid? (Procedures used to ensure data quality)
Simulated radar signals as described here, are reliant on the validity of the radar model being used.
Given that only trivial models are chosen, this risk is small, but not completely eliminated.
\todo{probably want a bit more here...}

% Describe the nature of the approach implementations
For the quantum methods to be implemented on simulated quantum hardware, an instruction set and simulator need be obtained. 
While several software packages fulfil this purpose, the Python-based quantum library \textit{Qiskit} \cite{qiskit_contributors_qiskit_2023} was chosen due to it being free, open-source, and generally familiar to the author and broader research community\cite{garhwal_quantum_2021}.
Furthermore, if it is desired that in future these methods be reproduced on real quantum machines, Qiskit has facilities for doing so.

Now that the empirical research framework has been established and the experimental objects, criteria, and materials have been described, each experimental method will be formulated.
