\section{Appendix}~\label{sec:appendix}
\subsection{Literature Review search terms}~\label{sec:appendix1}
\todojc{I am not sure if it is wise to put the search keywords as part of your introduction - perhaps if your research methods chapter identifies undertaking a lit review then put these there. Also, you may need to refine these as generic terms such as "quantum computing" could help you (only because you are a novice to QC) but are not focused on the problem at hand}
\begin{itemize}
    \item Quantum signal processing
    \item Quantum machine learning
    \item \ac{ESM} signal processing
    \item Temporal Quantum Computing
    \item Cognitive radar
    \item GNU Radio
    \item Qiskit
    \item Radar signal processing
    \item Pulse analysis radar
    \item Wavelet signal processing
\end{itemize}

\subsection{Literature evaluation questions}~\label{sec:appendix2}
As per research by Kitchenham et al. \cite{kitchenham_can_2010}
\begin{enumerate}[label=Q\arabic*]
    \item Do the authors clearly state the aims of the research?
    \item Do the authors describe the sample and experimental units
    \item Do the authors describe the design of the experiment?
    \item Do the authors describe the data collection procedures and define the measures?
    \item Do the authors define the data analysis procedures?
    \item Do the authors discuss potential experimenter bias?
    \item Do the authors discuss the limitations of their study?
    \item Do the authors state the findings clearly?
    \item Is there evidence that the Experiment/Quasi-Experiment can be used by other researchers/practitioners?
\end{enumerate}
