\section{Artefact Development Approach}~\label{sec:approach}

\subsection{signal encoding}

It may also be suggested that a frequency-domain input space be explored, in which case, the same experimental methodology should be applied, substituting IQ for a time varying frequency input. The reason for proposing a frequency-domain input signal is due to the general bandwidth limitation of practical IQ measurements as well as the more representative nature of frequency plots for radar signals.

% Quantum analysis algorithms:

The stages in the of development the artefact are:
\begin{quote}
    \textit{
        \begin{enumerate}
            \item Encode the input data into the state of a set of qubits.
            \item Bring the qubits into superposition over many states (i.e., use quantum superposition).
            \item Apply an algorithm (or oracle) simultaneously to all the states (i.e., use quantum entanglement amongst the qubits); at the end of this step, one of these states holds the correct answer.
            \item Amplify the probability of measuring the correct state (i.e., use quantum interference).
            \item Measure one or more qubits.
        \end{enumerate}
        - Quantum computing for finance: Overview and prospects: Román Orús, Samuel Mugeld, Enrique Lizaso \todo{Add this to citations}
    }
\end{quote}

\subsection{Frequency estimation}

instead of descretising signals into definite PDW's, the Quantum system might be able to generate a superposition of all possibilities of potential signal parameters at once. Further processing such as pulse detection and de-interleaving may be able to be conducted at once. This approach would investigate whether quantum methods can be used to place observed radar modes into a super-positioned and continuous state.