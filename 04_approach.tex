\section{Artefact Development Approach}~\label{sec:approach}

This section aims to describe the solution for each experiment: encoding, pulse detection, and frequency estimation.
First however, a description of the GNU radio functional block architecture: both the simulation layout, as well as how the quantum implementation is integrated.
A mathematical description of each method is also provided alongside a worked examples.

\subsubsection{GNU Radio block diagram}

Before the quantum solutions can be trialed, the simulation must be setup, and for this the test radars must be modelled in GNU radio.
Each radar was simulated in GNU radio according to the specifications outlined in the research methodology.
The most simple model being the \ac{CW} radar, which is a sinusoidal single signal source, and because of its triviality, no diagram is shown.
Similar to the \ac{CW} radar, the \ac{FMCW} radar is continuous, however, it has a modulation (\ac{FM}).
Figure \ref{fig:FMCW} shows the \ac{FMCW} radar model built in GNU radio.
It consists of a saw-tooth-shaped source which modulates the frequency of a sinusoidal source. A multiplier is added to set the amplitude of the output signal.
% 
\begin{figure}[ht]
    \centering
    \includegraphics[width=1\textwidth]{Figures/FMCW.png}
    \caption{\ac{FMCW} radar model}
    \label{fig:FMCW}
\end{figure}
% 
Finally, the pulsed \ac{CW} radar is one which produces regularly-spaced, non-coherent, un-modulated, sinusoidal pulses.
Figure \ref{fig:pulsed_CW} shows the radar modelled in GNU radio. It is a sinusoidal source multiplied by a vector souce which consists of a string of binary coefficients that, when multiplied with the other signal, produce regularly spaced puslses corrposponding to the input parametrers.
% 
\begin{figure}[ht]
    \centering
    \includegraphics[width=1\textwidth]{Figures/pulsed.png}
    \caption{Pulsed \ac{CW} radar model}
    \label{fig:pulsed_CW}
\end{figure}
% 

These models are then combined into a larger hierarchical block diagram (Figure \ref{fig:simulation})that creates a simulated radar environment.
The bottom left of the diagram consists of a number of radars - the specific types and parameters of which are detailed in Table \ref{tab:exp_definition}.
A noise source is also added to simulate environmental noise given some \ac{SNR}.
All these sources are then summed and filtered using a \ac{LPF} \ac{FIR} filter to cut higher frequencies which may skew any downstream frequency transforms.
Two additional components - the Frequency Sink and Time Sink - were used to debug the scenario, providing frequency-domain and time-domain plots respectively.
Finally, several Throttle blocks were added to ensure the number of samples per section does not exceed the defined sample rate.
The need for these arise due to there being no rate limited sinks that govern the velocity of samples through the system. Without them, the debugging displays incorrectly sample the signals and the file sinks become de-synchronised with each other.
% 
\begin{figure}[ht]
    \centering
    \includegraphics[width=1\textwidth]{Figures/simulation.png}
    \caption{Scenario simulation GNU radio block diagram}
    \label{fig:simulation}
\end{figure}


% ---------------- EXPERIMENT 1 ---------------- %
\subsection{Experiment 1}

The first experiment aims to encode a signal buffer - provisioned by the GNU radio simulation - into qubits.
Several methods are implemented: basis, encoding, and encodings.

Figure \ref{fig:encoding} outlines the block diagram used for simulation.
The signal comes from Figure \ref{fig:simulation} where it is generated, and subsequently passed directly into a custom GNU radio block - \lstinline{quantum_encode_sink}
The quantum encoding block contains the code for experiment 1.
It's purpose is to injest \lstinline{np.complex64} samples, encode them into a quantum circuit, decode them, and output the measurements as a \lstinline{np.float32}.
Throttles were again added to ensure all signals were synchronised with the sample rate.
\lstinline{quantum_encode_sink} was parameterised with number of qubits and shots (or number of measurements to take per sample) as seen in Figure \ref{fig:quantum_encode_sink_parameters}.
The second path originating from the simulator section is the Complex to Mag block. This simply evaluates the magnitude of each complex sample and passes it through to the time sink and file sink.
The time sink is used to visualise the measurements of the quantum encoding versus the raw samples, while the two file sinks save that data to two raw binary files (quantum.bin and raw.bin).
Finally, the Python snippet was written such that after the simulation is finished, it reads the two output files, transforms them into a more manageable CSV file, and displays a readout of the entire experiment in a line chart.
% 
\begin{figure}[ht]
    \centering
    \includegraphics[width=1\textwidth]{Figures/encoding.png}
    \caption{Test and experimentation block diagram}
    \label{fig:encoding}
\end{figure}
% 
\begin{figure}[ht]
    \centering
    \includegraphics[width=1\textwidth]{Figures/quantum_encode_sink_parameters.png}
    \caption{\lstinline{quantum_encode_sink} parameters}
    \label{fig:quantum_encode_sink_parameters}
\end{figure}

\textbf{Artefact}

The encoding was implemented in a single Python class of type GNU \lstinline{basic\_block}.
It uses several libraries - such as: \lstinline{logging, math, sys, Counter, matplotlib.pyplot, numpy, pywt, gnuradio}, and \lstinline{qiskit} - to achieve the encoding.
In the initialisation method, it sets up logging, selects the encoding method, and configures various parameters such as the number of qubits and shots.
The class implements the general\_work method, that is called by GNU Radio to process input and produce output.
The general\_work method retrieves the input buffer, initializes quantum and classical registers, performs the encoding based on the selected method, measures the quantum circuit, runs simulations, and retrieves the measurement results.
It has three selectable encoding options: 'basis', 'amplitude', and 'angle'.

According to the research methodology \ref{sec:design} in Table \ref{tab:exp_operation}, the evaluation of each encoding method is through: Measurement, circuit size, expressibility, sampling capacity, bandwidth, and computational efficiency.
Measurement, circuit size, and sampling capacity can be extracted from within the GNU radio, or the quantum encoding block via Qiskit, while the other criteria are analytically derived.
The evaluation is fully described in the evaluation \ref{sec:evaluation}.

\textbf{Methods}

The incident signal prior to sampling is a continuous complex-valued function
$x : \mathbb{R} \rightarrow \mathbb{C}, t \mapsto x(t)$,
where $x(t)$ represents the complex amplitude of the signal at any given time, $t \in \mathbb{R} > 0$.

The continuous signal is quantised,
$x : \mathbb{N} \rightarrow \mathbb{C}, t \mapsto x[t]$,
where $x[n] = x(n T_s)$, the $n^{th}$ sample of the signal, $n$ is an integer representing the sample index, and $T_s$ is the sampling period (i.e., $T_s = 1/f_s$).\footnote{$f_s$ is the sampling frequency of $300MHz$}

Let $\mathbf{x}$ be the buffer of samples $x_n=x[n]$.

The goal of encoding is to prepare the sample buffer for quantum processing

\textbf{Basis encoding}

The simplest of all non-trivial quantum encoding methods, basis encoding, maps a binary string $x \in {\{0,1\}}^n$ as coefficients of basis states.\footnote{In this paper,the convention of the computational basis is adopted, where the basis states are represented by the standard computational states $\ket{0}$ and $\ket{1}$.}

Let $\vert b_n \rangle$ be the $n^{th}$ orthonormal basis for a $N$-dimensional quantum system.
Then any state of the system can be written as a linear combination of the basis states:
\begin{equation}
    \displaystyle{
        \mathbb{R}^N \rightarrow | \mathbf{x} \rangle =
        \sum_{i=0}^{N-1}
            c_i | b_i \rangle
    }
\end{equation}
Where $c_i$ are complex coefficients of the basis states which satisfy the normalisation condition:
\begin{equation}
\label{eqn:normalisation_constraint}
    \displaystyle{\sum_{i=0}^{N-1} |c_i|^2 = \left\| \left|\psi\right\rangle \right\| = 1}
\end{equation}
% 
For basis encoding, the coefficients are further restricted such that $c_i \in \{0, k\}$, where $k \in \mathbb{R}^+$.
% 
The first step in this encoding is to define the mapping to a binary vector, $\textbf{b}$. If $\beta_i \in \{0, 1\}$ represents the $i^{th}$ bit of the transformed buffer, define $\textbf{b}$ as:
\begin{equation}
    \displaystyle{\mathbf{b} = \begin{bmatrix} \beta_0 \\ \beta_1 \\ \vdots \\ \beta_{2^{N}-1} \end{bmatrix}}
\end{equation}
% 
% Delta Encoding
% 
The transformation ($\textbf{x} \mapsto \textbf{b}$) will be a form of delta-encoding, wherein each element $d_i \in \mathbb{R}$ in array $\mathbf{D} = [d_0, d_1, \ldots, d_{2^N-1}]$ is defined by

\begin{equation}
    d_i = \lVert x[i] - x[i-1] \rVert
\end{equation}
% 
and first element defined by $d_0 = \lVert x[0] \rVert$.
% 
% Binary Encoding
% 
Given the delta array $\textbf{D}$, is real-valued and the desired transform is a binary array $\textbf{b}$, the next step proceeds to map each real element to a binary value of 1 for positive $d_i$, or 0 otherwise:
% 
\begin{equation}
    \beta_i = \begin{cases}
        1 & \text{if } d_i > 0 \\
        0 & \text{otherwise}
    \end{cases}
\end{equation}
% 
% Normalisation
% 
$\textbf{x} \mapsto \textbf{b}$, however, $\textbf{b}$ must satisfy the normalisation constraint (eqn. \ref{eqn:normalisation_constraint}).
This is done by dividing each element by the Euclidean norm, $\left\| \mathbf{b} \right\|$, yielding the normalised state vector $\ket{\psi (\textbf{x})}$.
% 
\begin{equation}
\displaystyle{
\ket{\psi (\textbf{x})} =
\frac{1}{\left\| \mathbf{b} \right\|}
\begin{bmatrix} \beta_0 \\ \beta_1 \\ \vdots \\ \beta_{2^N-1} \end{bmatrix}
}
\end{equation}
% 
Finally, the basis-encoded signal buffer is used to initialise the quantum circuit.
The implementation of which is delegated to Qiskit, using the \texttt{QuantumCircuit.initialize} method.

\textbf{Example Basis encoding}

%%%%%%%% example

% 
Given a signal buffer of four complex samples:
% 
\begin{equation}
\label{eqn:example_x_samples}
\mathbf{x} = \begin{bmatrix} 0.8 & 0.5 - 0.5j & 0.8 + 0.3j & 0.3 - 0.1j \end{bmatrix}^T
\end{equation}
% 
The magnitudes would be
% 
\begin{equation}
\mathbf{\lVert \textbf{x} \rVert} = \begin{bmatrix} 0.8 & 0.707 & 0.854 & 0.316 \end{bmatrix}^T
\end{equation}
% 
The delta array would be
% 
\begin{equation}
\mathbf{D} = \begin{bmatrix} 0.8 & -0.093 & 0.147 & -0.538 \end{bmatrix}^T
\end{equation}
% 
then, after binary decimation
% 
\begin{equation}
\mathbf{b} = \begin{bmatrix} 1 & 0 & 1 & 0 \end{bmatrix}^T
\end{equation}
% 
then normalising
% 
\begin{equation}
\ket{\psi (\textbf{x})} = \frac{1}{\sqrt{2}}\begin{bmatrix} 1 & 0 & 1 & 0 \end{bmatrix}^T
\end{equation}
The state would then be used to initialise a quantum circuit using Qiskit, 
\texttt{circuit.initialize(np.array(list(0.707, 0, 0.707, 0)), 2)}.
% 
The circuit would be
\[
\Qcircuit {
   \gate{\psi(\textbf{x})} & \qw & \gate{D} & \qw & \meter & \cw
}
\]
Where $D$ is a unitary which decodes the basis states encoded by $R$, for subsequent quantitative evaluation.

% % 
% where $\phi$ is the Qiskit state encoding. The measurement of this example would look like:
% % 
% \begin{equation}
%     \bra{\textbf{x}} = \ket{\psi (\textbf{x})}^\dagger = \begin{bmatrix}  \frac{1}{\sqrt{2}} & 0 &  \frac{1}{\sqrt{2}} & 0 \end{bmatrix}
% \end{equation}

\textbf{Amplitude encoding}

Amplitude encoding, similar to basis encoding, seeks to encode $\textbf{x}$ into coefficients of basis states.
However, the constraint of $c_i \in \{0, k\}$ is substituted by $c_i \in \mathbb{R}$, and therefore the binary delta transformation is extraneous.
As always, the normalisation constraint (eqn. \ref{eqn:normalisation_constraint}) must be upheld.
% 
The encoding is then simply a normalisation of the buffer:
% 
\begin{equation}
\displaystyle{
\ket{\psi (\textbf{x})} =
\frac{1}{\left\| \mathbf{x} \right\|}
\begin{bmatrix} x_0 \\ x_1 \\ \vdots \\ x_{2^N-1} \end{bmatrix}
}
\end{equation}

\textbf{Example amplitude encoding}

Using the same sample $\textbf{x}$ as in eqn. \ref{eqn:example_x_samples}, the amplitude-encoded state vector would be
% 
\begin{equation}
\mathbf{\ket{\psi (\textbf{x})}} = \frac{\textbf{x}}{\lVert \textbf{x} \rVert} =
\begin{bmatrix} 0.570 \\ 0.356 -0.356j \\ 0.570 +0.213j \\ 0.213 -0.071j \end{bmatrix}
\end{equation}
%
The circuit would identical to that of the basis encoding, but with the differently-encoded $\ket{\psi (\textbf{x})}$:

\[
\Qcircuit {
  \gate{\psi(\textbf{x})} & \qw & \meter & \cw \\
}
\]
% % 
% The measurement would be:
% % 
% \begin{equation}
% \bra{\textbf{x}} = \ket{\psi (\textbf{x})}^\dagger =  \begin{bmatrix} 0.8 & 0.707 & 0.854 & 0.316 \end{bmatrix}
% \end{equation}

\textbf{Angle encoding}

Angle encoding is a method which transforms the basis state $\ket{0}^{\otimes N}$ by rotating each qubit by two orthogonal angles, $\theta$ and $\phi$.

The two angles are computed by taking a single-level discrete wavelet transform ($\psi(\textbf{x}) \mapsto ([a_0, a_1, \dots], [d_0, d_1, \dots])$) of the input sample buffer, $\textbf{x}$.
$\theta_i$ is calculated from the $i^{th}$ approximation coefficient ($a_i$) and $\phi_i$ from the $i^{th}$ detail coefficient ($d_i$).

Each $a_i$ and $d_i$ are frequency-amplitudes, so to ensure a meaningful rotation, they must be normalised $\textbf{a}, \textbf{d}: \mathbb{R}^{2^N-1} \mapsto [0, \pi]^{2^N-1}$.
Since the number of qubits, $N$, is less than the length of the approximate and detail coefficient vectors, the $N$ most-recent coefficients are encoded into $\theta_0, \dots, \theta_N$ and $\phi_0, \dots, \phi_N$.

The rotation operation can be described using the formalism of quantum gates and unitary operators.
% 
The quantum gate that performs this rotation is called a general single-qubit rotation gate or a $R$ gate, which may be represented by a $2\times2$ unitary matrix:
% 
\begin{equation}
R(\theta, \phi) = 
\begin{bmatrix}
\cos(\frac{\theta}{2}) & -e^{i\phi}\sin(\frac{\theta}{2}) \\
e^{i\phi}\sin(\frac{\theta}{2}) & e^{i\phi}\cos(\frac{\theta}{2})
\end{bmatrix}
\end{equation}
% 
$\theta$ being the angle of rotation about the X-Y plane, and $\phi$ the angle of rotation about the Z-axis.
% 
The gate is be applied to each qubit $\ket{x_i}$, with the resulting state after the rotation being obtained by:
% 
\begin{equation}
\ket{x_i^\prime} = R(\theta, \phi) \ket{x_i}
\end{equation}
% 
The resultant circuit schematic is thus:
\[
\Qcircuit {
   & & & \mbox{} & & \\
   & \lstick{\ket{0}} & {/} \qw & \gate{R(\phi_i,\theta_i)} & \qw & \gate{D} & {/} \qw & \meter & \cw \\
   & & & \dstick{\psi(\textbf{x})} \cwx[-1] &
   \gategroup{2}{4}{3}{4}{.7em}{--}
}
\]
Where $D$ is a unitary which decodes the $R$ encoding by roation of the signal, for subsequent quantitative evaluation.

\textbf{Example Angle encoding}

Again, using the same sample buffer as in eqn. \ref{eqn:example_x_samples}, the encoding of angles proceeds as follows:

Discrete wavelet transform:
% 
\begin{equation}
    \psi(\textbf{x})
    \mapsto
    \begin{cases}
        \textbf{a} =
            \begin{bmatrix}0.759 & 0.589 \end{bmatrix} \\
        \textbf{d} =
            \begin{bmatrix}0.047 & 0.271\end{bmatrix}
    \end{cases}
\end{equation}
% 
$\textbf{a}$ and $\textbf{d}$ normalised yields:
\begin{equation}
    \textbf{a} =
        \pi\begin{bmatrix}0.790 & 0.613\end{bmatrix}
\end{equation}
\begin{equation}
    \textbf{d} =
        \pi\begin{bmatrix}0.171 & 0.985\end{bmatrix}
\end{equation}
%
The rotations matrices for $\textbf{a}$ and $\textbf{d}$ respectively are:
% A
\begin{equation}
R_{\textbf{a}}(2.481, 1.926) = 
\begin{bmatrix}
\cos(\frac{2.481}{2}) & -e^{i1.926}\sin(\frac{2.481}{2}) \\
e^{i1.926}\sin(\frac{2.481}{2}) & e^{i1.926}\cos(\frac{2.481}{2})
\end{bmatrix}
\end{equation}
% D
\begin{equation}
R_{\textbf{d}}(0.536, 3.10) = 
\begin{bmatrix}
\cos(\frac{0.536}{2}) & -e^{i3.10}\sin(\frac{0.536}{2}) \\
e^{i3.10}\sin(\frac{0.536}{2}) & e^{i3.10}\cos(\frac{0.536}{2})
\end{bmatrix}
\end{equation}

Then, taking applying these coefficients as rotations acting on each qubit, the encoding circuit unitary, $\textbf{R}$ is:
\begin{equation}
    \begin{bmatrix}
    -0.202+0.26j &  0.748-0.082j & -0.174-0.149j & 0.032+0.522 \\
    0.75 +0.058j &  0.21 +0.253j & -0.066+0.519j & -0.183+0.137 \\
    0.183+0.137j & -0.066-0.519j &  0.21 -0.253j & -0.75 +0.058 \\
    0.032-0.522j &  0.174-0.149j & -0.748-0.082j & -0.202-0.26 
    \end{bmatrix}
\end{equation}
% 
The rotation gate then acts on $\ket{0}^{\bigotimes 2}$, producing the encoded state vector:
\begin{equation}
    \ket{\psi (\textbf{x})}
    =
    \textbf{R} \cdot
    \begin{bmatrix}
        1 \\ 0 \\ 0 \\ 0
    \end{bmatrix}
    =
    \begin{bmatrix}
        -0.202+0.26j \\
        0.75 +0.058j \\
        0.183+0.137j \\
        0.032-0.522j
    \end{bmatrix}
\end{equation}
% % 
% and measured
% \begin{equation}
%     \bra{\textbf{x}} =
%     \ket{\psi (\textbf{x})}^\dagger =
%     \begin{bmatrix}
%         XXX &
%         XXX &
%         XXX &
%         XXX
%     \end{bmatrix}
% \end{equation}

% --------------
%%% Qualitative stuff / post hoc evals
% For although subsequent experimental trials and post hoc evaluation of the encoding methods remain as important means of assessing its suitability to the broader question, each experiment is designed to be independent of the results of the previous.
% In other words, while the success of the subsequent analyses hinges only somewhat on the quantitative comparison of encoding methods, it is the understanding of <...>. 
% Is is empirical trial and post hoc evaluation that measure this. 

% The method consists of creating a GNU-Radio block that permits the input of complex-valued samples into a buffer. The processing of samples is then done in Python and Qiskit within that block, and finally, outputted to a buffer where it is recorded.

\subsection{Experiments 2 and 3}

As per the research plan in \ref{sec:research_plan}, the planned process for continuing this research is to develop and test the artefacts for experiments 2 and 3 in trimester 2 of 2023.
The tentative approach for the development of these artefacts is as follows, however, the full description and explanation will be left until the next phase of the research project.
\begin{quote}
    \textit{
        \begin{enumerate}
            \item Encode the input data into the state of a set of qubits.
            \item Bring the qubits into superposition over many states (i.e., use quantum superposition).
            \item Apply an algorithm (or oracle) simultaneously to all the states (i.e., use quantum entanglement amongst the qubits); at the end of this step, one of these states holds the correct answer.
            \item Amplify the probability of measuring the correct state (i.e., use quantum interference).
            \item Measure one or more qubits.
        \end{enumerate}
        % - Quantum computing for finance: Overview and prospects: Román Orús, Samuel Mugeld, Enrique Lizaso \todo{Add this to citations}
    }
\end{quote}

Instead of quantising signals into definite \ac{PDW}'s, the Quantum system might be able to generate a superposition of all possibilities of potential signal parameters at once.
Further processing such as pulse detection and de-interleaving may be able to be conducted at once.
This approach would investigate whether quantum methods can be used to place observed radar modes into a super-positioned and continuous state.

In this section, the experimental test-bed artefact was presented, describing the constituent parts of the simulation, and the first of the three experiments was described. 
The experimental methods of basis, amplitude, and angle encoding have been developed for an application to \ac{ESM} signal processing.
Examples have been shown of each approach, and their subsequent evaluation and validation methods.
Finally the description of these artefacts demonstrates a major portion of the achievement of objectives 1-3, as well as a schematic plan of 4-7 to be undertaken in trimester 2.