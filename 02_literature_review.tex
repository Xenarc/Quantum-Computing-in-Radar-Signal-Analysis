\section{Literature Review}~\label{sec:literature}

\todojc{Instead of saying "Object Recognition and Identification Using ESM Data", use ref to the authors}Taghavi et al. \cite{taghavi_object_2016} present a survey of techniques in ESM and kinematic measurement, \todojc{rephrasing "and"}\textbf{as well as} a novel approach to the fusion of the two. 
The ultimate goal of their research was recognition and identification of \todojc{You may need to say what are "targets", e.g. "identification of objects of interests in radar signals, to be later used for military targets", or whatever is appropriate}\textbf{targets}. 
The authors present a novel architecture for combining both \todojc{"approaches"?}\textbf{data-sets}, suggesting that traditional approaches or silo-ed analyses are limited. 
In the paper, they stress the importance of using orthogonal source data; that is, the inputs of the system must represent non-overlapping phenomena. 

While their \todojc{You've never mentioned "problem solving"}\textbf{problem solving} methodology was novel, \todojc{What "execution"? I thought you said it was a survey? You may need to say they have also provided algorithms for different tasks? Or that the survey was limited to maritime applications}\textbf{the execution} was somewhat limited in the respects that they restricted their simulation to the maritime domain, and with simulated data. If their methods were to be applied practically, the limitation imposed by the maritime domain implicitly reduces the possibility for multi-path ambiguities. The small amount of real-world data also show a systemic challenge in radar signal analysis research. Furthermore, while their survey presented several classical approaches for the estimation of source parameters (both ESM and kinematic), the methods were only analytical. There was no mention of machine learning-based recognition and identification frameworks. The analytical methods presented were generally applicable and relevant for a baseline of ESM techniques. For example: Bayesian framework, belief functions, fuzzy set theory. The data-fusion framework presented in the paper does provide an applicable architecture for multi-domain analysis which may prove useful in future research. % Maybe I should make this more relevant to the problem at hand? Like drawing parralels to the challenges and approaches suggested for our research. % On second thought, each such analytical method should be explained 

\todojc{You do not need to provide the titles, e.g. "Deep-Learning for Radar: A Survey", also note that we say "et al." as it is abbreviation from "et alia", and the following is plural}Geng \textbf{et al.} \cite{mason_deep_2017} \textbf{provide} a broad outlook on the current state of affairs in the use of deep learning in radar signal processing. The authors analysed a large body of research over several areas of radar signal processing and for a multitude of deep learning techniques. The paper identifies numerous applicable approaches with varying degrees of effectiveness. Deep learning techniques discussed were:
\begin{itemize}
    \item CNN: Convolutional neural network
    \item DBN: Deep belief network
    \item DRL: Deep reinforcement learning
    \item DRES: Deep residual neural network
    \item FNN: Feed-forward neural network
    \item GAN: Generative adversarial network
    \item RNN: Recurrent neural network, including long-short term memory
    \item SAE: Stacked auto-encoder
\end{itemize}

The methods analysed in the paper, while comprehensive overall, still failed to attack the problem of parameter estimation. 

Several limitations were also identified, some being systemic to the field, others specific to deep learning techniques. Some challenges identified were their subjectivity to Electronic Attack (EA), where, depending on the level of intelligence about the Deep Neural Networks (DNN) techniques employed, various exploitations may be executed. Of course, EA is not specific to DNN's, however, it's noted that because they exist as an heuristic model, they neccisarily have gaps in their model of the problem space. Therefore, it's noted that DNN's may be more suseptable to EA because of its inherent non-analytical method. Similar to Taghavi et. al. the lack of training data is a serious challenge in the testing and validation of any radar analysis model. This proves a large problem in any future effor for open-source development of DNN's for radar analysis, as noted by the authors that DNN's require large amounts of training data. The paper concludes with an evaluation of particularly effective methods, some reaching > 97\% accuracy, however as described - with small training sample sizes.

The primary interest of this option is in passive radar, i.e., radars which generally do not transmit energy and listen to a return. An example is a Radar Warning Receiver (RWR) which is a passive electronic warfare support system \cite{avionics_department_electronic_2013}

One functional mode of RWR's is the search mode: where the system’s objective is to interrogate the electromagnetic environment to detect and identify emitters. The only information that radars are given is the signals which it receives, and its location in space and time (as well as some prior understanding of the operational environment). Due to the nature of the operating context, modern radars may receive many incident signals in a short period of time. Within such received signals, there may exists a number of constituent parts: communications signals, radar signals, interference (whether intentional or not), and noise (being environmental, or system). For signals originating from radars, they vary in several dimensions: see radar domains. Noise is generally constant, but limits the receiver’s ability to detect a given signal’s presence. Interference is any signal which interferes with the receiver’s ability to perform its function – either fully or partially. The functional operation of the receiver is: signal input (coming from antenna), RF front-end, digitisation, and processing. In each stage, the signal may change characteristics and formats based on a variety of filtering and processing operations. \cite{stimson_introduction_1998}\todo{This last one needs to be refined and specified to more constraints / trade-offs of radar I think.}

"Encoding patterns for quantum algorithms" - Weigold et.al. \cite{weigold_encoding_2021} details several applicable approaches to encoding data for quantum algorithms. The problem presented in the paper is that quantum computing may have (in some cases) exponential speed-up when compared to classical methods, however data loading is often inefficient. Starting with a description of the quantum format itself, (qubits), the paper implicitly shows how quantum methods differ from classical formats through an explanation of the former. The substance of the article then goes on to focuses on a comparison of practical means of transforming data, specifically:
\begin{itemize}
    \item Basis encoding
    \item Angle encoding
    \item QuAM encoding 
    \item QRAM encoding
    \item Amplitude encoding
\end{itemize}

The paper comprehensively details each approach, their applications, methods, and variants. Interestingly, the authors also present four novel methods on quantum state preparation, such that once data are loaded from their classical sources, that they may be prepared for quantum operations. The approaches were covered:

\begin{itemize}
    \item Schmidt decomposition
    \item Matrix encoding
    \item Quantum Phase Estimation
    \item Post-selective measurement
\end{itemize}

Finally, an example solution to the well known quantum algorithm for linear systems of equations - the HHL algorithm - using several of the mentioned approaches were compared.

Overall, the paper makes a comprehensive survey of encoding approaches with strong analytical grounding. The approaches mentioned, while soundly applicable to many forms of atomic classical data, somewhat neglect the complexity of real-world data. A specific example would be high-dimensional timeseries (as is used for radar signal processing), which neither classical computing nor quantum computing have yet to represent faithfully. Such a limitation may be simply a result of the technologies used, rather than the methods presented in this article. The paper effectively describes and identifies the benefits and limitations of each method in terms of efficiency and entropy conversion trade-offs.
\todojc{I agree with your hidden notes on the need for more synthetic lit review}

% I don't think I've hit the mark here...
% There needs to be more synthetic analysis here. Comparisons between each approach and the subject of this report need to be drawn.
% I think these reviews should be shorter, more relevant, and more numerous. A bit like the background, but I don't want to step on my own toes by repeating that information.
% I'm also not so sure about how detailed the review should be? Should I be describing the methods used in each, or just a broad overview as what I have done here?
% To what degree do the papers have to be 'current' seeing that most research is exploring current paradigms using more specific techniques, whereas the subject of this report is more 'from first principles'
% On a similar note, to what extent is this explanation versus critical analysis? 

% NOTES FROM MEETING 1
% 1. Academic references - explain / summarise them\\
% 2. Develop a meta-model of their fundamental argument, challenges, deficiencies, and strengths\\
% 3. Compare their differences, similarities, and whether or not they agree. (If they all agree, more sources should be found); how they interrelate.\\
% 4. Next, the exposition of the problem space (radar signal analysis) should be written, being lean enough so as to only explain the content of the referenced academic sources. I.e., the challenges faced should be fully developed.\\
% 5. Following which, a more precise formulation of problems should be undertaken.

% Here are some ones I would like to properly analyse in more depth (not yet added to Zetaro)
% - https://ietresearch.onlinelibrary.wiley.com/doi/pdf/10.1049/iet-rsn.2017.0516
% - https://ietresearch.onlinelibrary.wiley.com/doi/pdf/10.1049/iet-rsn.2017.0563
% - Improvements on deinterleaving of radar pulses in dynamically varying signal environments: Kenan Gençol, Ali Kara, Nuray At (2017)
% - Estimating the Instantaneous Frequency of Linear and Nonlinear Frequency Modulated Radar Signals — A Comparative Study Hubert Milczarek, Czesław Le´snik, Igor Djurovi, and Adam Kawalec

% Perhaps a more thematic structure would suit:
% I. Comparison of classical processing and quantum processing, the foundations of each, their challenges, and broad evaluation 
% II. Survey of the causation / principles of existing methods of radar signal processing.
% III. Survey of the of quantum methods of processing, (perhaps an explanation of them?)
% IV. Emerging / novel approaches - specifically link to future research (such as this)

