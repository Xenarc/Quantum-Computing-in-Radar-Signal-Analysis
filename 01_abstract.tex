\section*{Abstract}

This report explores the potential application of quantum computing methods in radar signal analysis for \ac{EW}.
With traditional signal processing techniques nearing the edge of classical computing's extent, new approaches are required to keep up with emerging threats.
Much research has been conducted on quantum computing and it's potential applications to signal processing, however, little work has been done in its application to radar signal processing for \ac{EW}.
This report presents a survey of the current state of research, as well as an exploratory probe into how quantum methods can encode, detect, and characterise radar signals.
Novel quantum methods for encoding signals, determining radar pulse boundaries, and frequency characterisation will be presented.\todojc{I think your abstract should end here, the rest is for the introduction}
The milestones for the project are:
\begin{enumerate}
    \item Acquire a signal simulator; generate source data with valid radar models
    \item Identify, and implement several quantum-based encoding methods that convert sampled time-domain radar signals into a form which prepares them for further quantum processing.
    \item Develop a method for quantum-based pulse detection of a radar signals. Test the method by varying the type and quality of radar signals. Present quantifiably measurable results.
    \item Determine a quantum method for estimating a radar signal’s frequency. Test the algorithm on a variety of radar signals and present quantitative results of its performance.
\end{enumerate}

The research plan is, in the first phase in T1, to conduct experiment 1 - encoding. In the second phase - T2, to continue with executing experiments 2 and 3.